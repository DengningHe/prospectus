\documentclass{proc}

\begin{document}

\title{Establishing Sea-Based Trade Routes to the East\\for the Spanish Empire}

\author{Remco Chang, Christopher Columbus}

\maketitle

\section{Abstract / Introduction}

The recent development of trading with markets to the East has provided significant potential opportunities for individuals and governments in European countries to gain access to novel new products, as well as to grow economically.
Based upon the establishment of viable trade routes to the Orient, pioneered by the Italian, Marco Polo and his colleagues, citizens of Europe are increasingly demanding the rich silk textiles, exotic teas, spices, and other food products from the East.
As a consequence, such goods demand premier prices, and forward-thinking entrepreneurs and governments can be expected to gain great wealth by capitalizing upon these opportunities to gain significant market share.

Unfortunately, established trade routes to the East are currently extremely perilous and involve treacherous mountain passes, dangers river crossings and threats from armed bandits.
As a consequence, only 15\% of merchandise intended for European markets from the East actually reaches its destination.
In addition, the risk to the lives of those involved in these trade routes is exceptionally high, and it is becoming increasingly difficult to recruit merchants willing to commit resources to such ventures.
As a consequence, there exists a critical need to identify alternative trade routes that can be used to more cost-effectively and efficiently bring goods and merchandise from the east to markets in Central Europe. 
In the absence of such alternative strategies, trade with the East will continue to be a challenging and economically-risky practice. 

In this paper, we adopt the hypothesis by Ehrmantraut et al. [1] that the Earth is indeed round. 
With this assumption, we demonstrate that by amassing a giant fleet and continuing going west, we will eventually reach the Orient and the Far East.
Through simulation, we show that this route is just as fast as the land route, and is much safer (assuming that there are no sea monsters).
We conclude by discussing the potential to the Spanish Empire by adopting this approach to establish new sea-based trade routes to the East.

\section{Describe your project in one sentence}

Based on the assumption that the Earth is round, we demonstrate through simulations that a sea-route to the Far East and the Orient is at least as fast and safe as the traditional land-based route.

\section{What type of project is this and why?}

Simulations / Application

\section{Who is the audience for this project? How does it meet their needs? What happens if their needs remain unmet?}

The outcomes of this project will have great impact to both the academic community as well as the government policy makers.
As the community searches for novel ways to reach the Far East and the Orient more quickly and safely, the results of this work can open a new possibility that will have great national impact.

Without adopting the proposed sea-based approach, the Spanish Empire will continually need to rely on Central European countries for silk, spices, and textiles.
In the long term, this will directly affect the international position of the Spanish Empire as a super power of the world.

\section{What is your approach and why do you think it’s cool and will be successful?}

\begin{enumerate}
\item The recent theory that the Earth is round by Ehrmantraut et al. [1] suggests that sailing towards the west will eventually reach the Far East.
\item Advances in simulation techniques, in particular the SimWorld system developed by Fring et al. [2] is fast, accurate, free to download, and open source.
\item Various sets of possible parameters of the model have been published by White et al. [3] and Pinkman et al. [4]. Both approaches are appropriate for our needs, we will investigate which one will give us the best results.
\end{enumerate}


\section{In the best-case scenario, what would be the impact statement (conclusion statement) for this project?  } 

The good scenario is that the simulation results support our hypothesis that a sea-based trade route is as fast and safe as the land-based route.  The best case scenario is if the sea-route is faster by at least 30\% and is also more cost effective.


\section{List all major milestones for this project.}

\begin{enumerate}
  \item Download and install the SimWorld package
  \item Experiment with the two sets of parameters by White et al. [3] and Pinkman et al. [4]
  \item Run 100 simulations with different random seeds
  \item Analyze the results using R
  \item Critically compare the results with the published land-based trade route results by Saul Goodman [5]
\end{enumerate}

\section{What obstacles do you anticipate?}

\subsection{Major obstacles} 

\begin{itemize}
  \item There are rumors that SimWorld is finicky about running in the Windows environment, but SimWorld hs stopped supporting Linux just recently.  If we can't get SimWorld to run correctly, we will need to invest a large amount of time to port the system to Linux ourselves.
  \item The parameters published by White et al. and Pinkman et al. were based on the tourism industry. There is a possibility that their results might not be appropriate for our needs.  If that's the case, we are sunk.  We cannot afford to spend the time to validate a set of parameters ourselves.
\end{itemize}

\subsection{Minor obstacles}

\begin{itemize}
  \item It is possible that the computational resources required by SimWorld exceeds those available to us here at Tufts.  If that's the case, we will need to run the simulations on the VisWall servers.
  \item We might need another person to help interpret the results of the simulations and compare them to the results by Saul Goodman. If we can't find this person from the Geography department, we will have to make some guesses ourselves. This could hurt the value of our work.
\end{itemize}

\section{What additional resources do you need to complete this project?}

\begin{itemize}
  \item Expertise in understanding the results from SimWorld.
  \item Some help with using R for analyzing this type of complex simulation results.
  \item Someone to give us access to the VisWall servers.
 \end{itemize}

\section{List 5 major publications that are most relevant to this project, and how they are related.}

\begin{itemize}
\item As noted earlier, the works by White et al. [3] and Pinkman et al. [4] are crucially related, but their context is in the tourism industry.

\item Saul Goodman's results [5] on land-based trade route is certainly relevant as we will use them as a baseline in our comparison.

\item Since the entire project hinges on the assumption that the Earth is round, we will need to cite the work by Ehrmantraut et al. [1], as well as some less known work by Schrader and Schrader [6].  Additional literature review in this area will be helpful.

\item In using and analyzing the results of SimWorld simulations, we will need to look at the work by Hector Salamanca [7], who is an associate of Gustavo Fring, the developer of SimWorld.

\item Finally, it will be relevant to note the recent parallel efforts by Ferdinand Magellan [8] and Juan Sebastián Elcano [9].  Their goals are to circumnavigate the Earth, but their efforts will share synergistic activities with our work.
 \end{itemize}


\section{When / How do you know if you have succeeded in this project?}

I think that this research will be publishable if we can show that the sea-based route is \emph{comparable} to the land-based route.  Although I can't say this for sure, but even if the sea-based route is slower, less safe, and more costly, this could still be publishable as long as it's not slower, less safe, or more costly by more than 20\%.

%\bibliographystyle{abbrv}
\end{document}
